\documentclass[11pt,paper=a4,final]{scrartcl}
\usepackage[utf8]{inputenc}
\usepackage{geometry}           %allows us to specify the 'seitenrand'
\usepackage{graphicx}           %package used to include graphics
\usepackage{hyperref}           %used to make klickable links
\usepackage{listings}
\usepackage{tabularx}
\usepackage{pdflscape}
\usepackage[figuresright]{rotating}
\usepackage{nameref}
\usepackage{longtable}
\usepackage{enumitem}
\usepackage{ngerman} % Make the document German
% Make the document German
\usepackage{ngerman}
\usepackage{fancyhdr}
\usepackage{lipsum}
\usepackage{mdwlist}
\usepackage{multirow}

\hypersetup{
    colorlinks,
    citecolor=black,
    filecolor=black,
    linkcolor=black,
    urlcolor=black
}


\title{Raport}
\author{Niklaus Hofer}
\date{\today{}}

% Make title and author accessible in the header/footer
%\makeatletter
%  \let\Title\@title
%  \let\Author\@author
%\makeatother
%
\pagestyle{fancy}
%
\geometry{a4paper, top=20mm, right=20mm, bottom=20mm, left=20mm}
%
%\fancyhf{}      %delete default values
%\setlength{\headwidth}{\textwidth}      %header and footer width equal the text width
%\lhead{\Author}
%\rhead{\Title}
%\fancyfoot[CE,CO]{Speicherdatum: \today{}}
%\fancyfoot[RE,RO]{\thepage}

\begin{document}
\begin{landscape}
  \begin{longtable}{|p{1.8cm}|p{1.5cm}|p{5.0cm}|p{11.0cm}|l|l|}
    \hline
    \multirow{2}{*}{\bf Datum} & \multirow{2}{*}{\bf Wer} &\multirow{2}{*}{\bf T\"atigkeit} & \multirow{2}{*}{\bf Reflexion} & \multicolumn{2}{c|}{\bf Zeit} \\ \cline{5-6}
     & & & & \bf Geplant & \bf Effektiv \\ \hline
    %Use the headings below if you don't have multirow installed
    %\bf Datum & \bf Wer & \bf T\"atigkeit & \bf Reflexion & \bf Geplant & \bf Effektiv \\ \hline
    \hline
    \endhead
    % Beginn section ---------------------------------------------------------
    2012.11.23 & Niklaus Hofer &
    Besprechung mit Herr Tschopp \"uber den Projektstand und das geplante weitere Vorgehen. &
    Bis jetzt siehts nicht schlecht aus. Wir sind gut im Zeitplan. Allerdings sollten wir mit dem Erstellen des Wikipedia-Artikels beginnen. Wir haben abgemacht, dass ein Entwurf der Struktur des Artikels anfangs Dezember vorliegen sollte. Um diese Struktur zu erstellen, werden wir andere Wikipedia-Artikel zu \"ahnlichen Themen auf ihren Aufbau untersuche. Die Resultate werden wir festhalten und uns dann beim Erstellen des eigenen Artikels daran orientieren.&
    40min & 20min \\ \hline \nopagebreak
    \multicolumn{2}{|l|}{\bf Pendenzen} &\multicolumn{2}{p{16.0cm}|}{
      \begin{itemize}
        \item Beginn der Arbeiten am Wikipedia-Artikel
	\begin{itemize}
	  \item Erstellen des Rasters zum Untersuchen der Wikipedia-Artikel
	  \item Wikipediaartikel zu \"ahnlichen Themen untersuchen
	  \item Vorlage f\"ur den eigenen Artikel erstellen
	 \end{itemize}
	 \item neuen Termin mit Herr L\"uthi vereinbaren
      \end{itemize}
    }  & \multicolumn{2}{l|}{} \\ \hline
  \end{longtable}
\end{landscape}
\end{document}
