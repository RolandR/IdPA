\documentclass[11pt,paper=a4,final]{scrartcl}
\usepackage[utf8]{inputenc}
\usepackage{geometry}           %allows us to specify the 'seitenrand'
\usepackage{graphicx}           %package used to include graphics
\usepackage{hyperref}           %used to make klickable links
\usepackage{listings}
\usepackage{tabularx}
\usepackage{pdflscape}
\usepackage[figuresright]{rotating}
\usepackage{nameref}
\usepackage{longtable}
\usepackage{multirow}

\hypersetup{
    colorlinks,
    citecolor=black,
    filecolor=black,
    linkcolor=black,
    urlcolor=black
}

\usepackage{fancyhdr}
\pagestyle{fancy}
% \setlength{\parskip}{0pt}
% \setlength{\baselineskip}{0pt}
\parindent 0pt 
\parskip 11pt
\parsep 0pt 
\itemsep 0pt 
\topsep 0pt 

\geometry{a4paper, top=20mm, right=15mm, bottom=25mm, left=25mm}

%defining header and footer
\fancyhf{}      %delete default values
\setlength{\headwidth}{\textwidth}      %header and footer width equal the text width
%\fancyhead[LE,LO]{\includegraphics[scale=0.6]{header.png}}
\fancyhead[LE,LO]{Niklaus Hofer und Roland Rytz}
\fancyhead[RE,RO]{Idpa}
\fancyfoot[CE,CO]{Speicherdatum: \today{}}
\fancyfoot[RE,RO]{\thepage}

%New page for every section
\let\stdsection\section
\renewcommand\section{\newpage\stdsection}

\title{Arbeitsjournal}
\author{Roland Rytz, Niklaus Hofer}
\date{\today{}}

\begin{document}

\maketitle
\vfill
\begin{flushleft}
Gewerblich Industrielle Berufsschule Bern
\end{flushleft}
\newpage
\begin{landscape}
\section{Arbeitsjournal}
Datum im Format Jahr.Monat.Tag
\begin{longtable}{|p{1.8cm}|p{1.5cm}|p{7.5cm}|p{9.0cm}|l|l|}
\hline
\multirow{2}{*}{\bf Datum} & \multirow{2}{*}{\bf Wer} &\multirow{2}{*}{\bf T\"atigkeit} & \multirow{2}{*}{\bf Reflexion} & \multicolumn{2}{c|}{\bf Zeit} \\ \cline{5-6}
 & & & & \bf Geplant & \bf Effektiv \\ \hline
%Use the headings below if you don't have multirow installed
%\bf Datum & \bf Wer & \bf T\"atigkeit & \bf Reflexion & \bf Geplant & \bf Effektiv \\ \hline
\hline
\endhead
2012.09.09 & Niklaus und Roland &
Initialisierung des Repositories. &
Die Zusammenarbeit funktioniert bis jetzt gut. &
20min & 20min \\ \hline \nopagebreak
\multicolumn{2}{|l|}{\bf Pendenzen} &\multicolumn{2}{p{16.5cm}|}{Planen des weiteren Vorgehens.}  & \multicolumn{2}{l|}{} \\ \hline
\end{longtable}
\end{landscape}
\newpage
%\bibliographystyle{plain}
%\bibliography{documentation}{}
\newpage
\section{Erkl\"arung}
Wir best\"atigen, die vorliegende Arbeit selbst\"andig verfasst zu haben. S\"amtliche Textstellen, die nicht von uns stammen, sind als Zitate gekennzeichnet und mit dem genauen Hinweis auf ihre Herkunft versehen. Die verwendeten Quellen (gilt auch f\"ur Abbildungen, Grafiken u.\"a.) sind im Literaturverzeichnis aufgef\"uhrt.

\vspace{4cm}
Bern, \today{}
\vspace{4cm}

Niklaus Hofer\hfill Roland Rytz\\
\vspace{2cm}
\hrulefill \hfill \hrulefill
\newpage
\end{document}
